%-----------------------------------------------------------------------
%	SECTION TITLE
%-----------------------------------------------------------------------
\cvsection{{Details of Research Experience}}

%-----------------------------------------------------------------------
%	SECTION CONTENT
%-----------------------------------------------------------------------
\begin{cvinterests}

%------------------------------------------------

\cvinterest
{
\textbf{Exotics searches}\\
\vspace{-1.8em}
\begin{itemize}[labelwidth=0.05in,align=right,leftmargin=!,labelsep=0pt,
itemsep=0.0em]
\item[] \textsc{Search for new possible heavy neutral lepton-flavor violating
(LFV) dilepton final states}\\
Very strong contribution in the search for new possible heavy neutral
lepton-flavor violating dilepton final states, as analysis contact, editor
of publications and main analyser. A lepton-flavour violating decay would be a
clear indication of new physics. Moreover, processes with flavor-violating
dilepton decays have a clear experimental signature and a low SM background.
Given this, the easiest way to look for such process is to search into a high
mass regime and look for events with one different-flavor lepton pair
($e\mu$, $e\tau$, $\mu\tau$) in the final state. Different Beyond Standard
Model (BSM) models have been considered in this search: the LFV $Z^{\prime}$,
the R-parity Violation (RPV) SuperSymmetric (SUSY) $\tilde{\nu}_{t}$, both
predicting a high mass resonance, and Quantum Black Hole (QBH) production.\\
The first round of this search has been published using
\SI{3.2}{\per\femto\barn} of $pp$ collision data at 13 \si{\tera\electronvolt}
considering only electrons and muons in the final state
(\href{https://atlas.web.cern.ch/Atlas/GROUPS/PHYSICS/CONFNOTES/ATLAS-CONF-2015-072}{ATLAS-CONF-2015-072}).
A second round has been published using the same amount of $pp$ data and
including also taus in the final state (\href{https://link.springer.com/article/10.1140/epjc/s10052-016-4385-1}{Eur. Phys. J. C 76 (2016) 541}).
The last round aims at publishing a paper using \SI{36.1}{\per\femto\barn} of
$pp$ collision data recorded between 2015 and 2016: the paper has been already submitted to
PRL journal (\href{https://arxiv.org/abs/1807.06573}{arXiv:1807.06573 [hep-ex]}).\\
For this search I developed the official analysis framework, which provides the
main object and event selection and the SM background processes estimation:
it is being used since Summer 2015. I optimised the event
selection in order to get the highest background rejection and signal
significance. I also studied the background processes arising from jets
mis-reconstructed as leptons ($W$+jets and multi-jet processes), which have a
key role in final states with taus, being up to 50\% of the total expected SM
background. Moreover I evaluated the relevant systematic uncertainties
(theoretical and experimental ones) on the background estimates.
%
\item[] \textsc{Search for new possible heavy neutral gauge bosons decaying in
muon pairs}\\
During my master thesis I studied the discovery potential of the ATLAS
experiment and the production and decay properties of a possible new gauge
heavy boson, $Z^{\prime}$, decaying in a pair of muons. In order to understand
the detector capability in reconstructing such decays, I studied the
performance of the ATLAS tracking system, such as the muon momentum resolution
and tracking and trigger efficiencies, highlighting a large decrease of
efficiency in the presence of strong detector misalignments. Such effect, due to
the failure in matching Inner Detector (ID) and Muon Spectrometer (MS) tracks,
was fixed in the main muon reconstruction code and reported in an ATLAS internal
note (\href{https://cds.cern.ch/record/1191139}{ATL-PHYS-INT-2009-067}).
The ATLAS discovery potential has been studied focusing on the $Z^{\prime}$ boson
predicted by a simple extension of the SM - the Sequential Standard Model (SSM).
The statistical method of a binned likelihood ratio was used for this study.
\end{itemize}
%
\textbf{Standard Model Higgs boson search and property measurements}\\
\vspace{-1.8em}
\begin{itemize}[labelwidth=0.05in,align=right,leftmargin=!,labelsep=0pt,
itemsep=0.0em]
\item[] \textsc{Higgs boson search and discovery}\\
Strong contribution to the Higgs boson search analysis in the the four lepton
decay channel.\\
In the beginning of 2012, I strongly contributed to the kinematic selection
optimisation, fine tuned in the 120--130 \si{\giga\electronvolt} mass region.
The optimisation has been done in order to increase the signal-background
discrimination power and to reach a higher sensitivity to Higgs boson signal
events. The kinematic variables considered were the opposite-charge dilepton
invariant mass closer to the PDG $Z$ mass, the other opposite-charge dilepton
invariant mass and the leptons transverse momenta. I contributed to the
background estimation of the irreducible background, verifying the SM
prediction of continuum $ZZ$ process through dedicated studies and using
different MC generators. I also contributed to the reducible background
estimation ($Z$+jets and $t\bar{t}$), performed in control regions where the
$b\bar{b}$ or $t\bar{t}$ contributions are enhanced
(\href{https://atlas.web.cern.ch/Atlas/GROUPS/PHYSICS/CONFNOTES/ATLAS-CONF-2012-092/}{ATLAS-CONF-2012-092},
\href{http://www.sciencedirect.com/science/article/pii/S037026931200857X}{Phys. Lett. B 716 (2012) 1-29}).
%
\item[] \textsc{Higgs boson property measurement}\\
Key contribution to the Higgs boson spin/CP measurement and participation
to the first Higgs boson mass and couplings measurement in the context of the
four lepton decay channel.\\
After the Higgs boson discovery I started the spin and parity properties
measurement. I produced and validated the Monte Carlo signal samples under
different spin and parity hypotheses. The final state of the ``golden channel''
is well defined from the four reconstructed leptons: this means that the
spin/parity properties can be exploited using the angular and kinematic
distributions. I developed a machine learning algorithm based on the Toolkit
for Multivariate Analysis (TMVA) with the Boosted Decision Tree (BDT) algorithm.
The separation between states with different spin
and parity has been achieved by using a multivariate discriminant, built for
each spin/parity hypothesis using seven sensitive variables to the underlying
spin and parity: the masses of the $Z$ bosons, one production angle and four
decay angles. By using only the events in the $m_{4l}$ window 115-130
\si{\giga\electronvolt}, the exclusion of a spin/parity hypothesis was
achieved by comparing the response shape of the BDT discriminants calculated
for the signal and background samples to those observed in data
(\href{https://atlas.web.cern.ch/Atlas/GROUPS/PHYSICS/CONFNOTES/ATLAS-CONF-2012-169/}{ATLAS-CONF-2012-169}, 
\href{http://www.sciencedirect.com/science/article/pii/S0370269313006369}{Phys. Lett. B 726 (2013), pp. 88-119}).\\
At the end of the RUNI data taking, the whole analysis has been updated using
the full RUNI $pp$ recorded data (\SI{25}{\per\femto\barn}). I performed a
further optimisation of the selection criteria, such as the lepton quadruplet
selection criteria, in order to reduce the amount of background events and
increase the signal significance. Given the Higgs boson spin/CP nature, I
strongly contributed to the further improvement to the signal versus $ZZ*$
background discrimination in order to increase the resolution of the mass
measurement. This was done developing a multivariate kinematic discriminant,
based on the $p_{T}$ and $\eta$ of the four-lepton final state and the
Leading Order Matrix Element information, taking the advantage of the spin/CP
hypothesis of the Higgs boson. Moreover, I contributed to the development of
a new multi-dimensional fit procedure to extract the mass value of the Higgs
boson and, consequently, the couplings
(\href{https://atlas.web.cern.ch/Atlas/GROUPS/PHYSICS/CONFNOTES/ATLAS-CONF-2013-013/}{ATLAS-CONF-2013-013},
\href{http://journals.aps.org/prd/abstract/10.1103/PhysRevD.90.052004}{Phys. Rev. D 90, 052004 (2014)},
\href{https://journals.aps.org/prd/abstract/10.1103/PhysRevD.91.012006}{Phys. Rev. D 91, 012006 (2015)}).
%
\item[] \textsc{Tensor structure and Higgs boson Spin/CP measurement update}\\
Contribution in cross-checking the results by developing an independent code
for toys generation and in editing the supporting documentation for the
published paper.\\
This study has been done using the full LHC RUNI LHC data and the three bosonic
channels: $H\to\gamma\gamma$, $H\to ZZ^*\to4l$ and $H\to WW^*\to l\nu l\nu$.
Concerning the spin measurement, several alternative models have been included
for the spin 2 hypothesis, with universal and non-universal couplings to
fermions and bosons, with respect to the previous published analysis
(\href{http://www.sciencedirect.com/science/article/pii/S0370269313006369}{Phys. Lett. B 726 (2013), pp. 88-119}).
Moreover, the tensor structure of the HVV interaction in the spin-0
hypothesis has been investigated by using the $H\to ZZ^*\to4l$ and
$H\to WW^*\to l\nu l\nu$ decay modes
(\href{https://atlas.web.cern.ch/Atlas/GROUPS/PHYSICS/CONFNOTES/ATLAS-CONF-2015-008/}{ATLAS-CONF-2015-008},
\href{https://link.springer.com/article/10.1140/epjc/s10052-015-3685-1}{Eur. Phys. J. C75 (2015) 476}).
%
\item[] \textsc{Higgs boson fiducial and differential cross section
measurement}:\\
Strong contribution to the Higgs boson fiducial and differential cross section
measurement in the context of the four lepton decay channel.\\
During the LHC RUNII data taking, the analysis aiming at measuring the Higgs
boson cross section has been performed in several rounds, each one published or
aiming to be published as CONF note or paper. My contribution started in Spring
2016, when I performed the cross-check analysis of the fiducial cross section
measurement published using \SI{14.8}{\per\femto\barn} of 13 \si{\tera\electronvolt}
data as a CONF note (\href{https://atlas.web.cern.ch/Atlas/GROUPS/PHYSICS/CONFNOTES/ATLAS-CONF-2016-079/}{ATLAS-CONF-2016-079}).
I then became one of the main analyser and editor of the supporting
documentation in the next analysis round. One of the main goal of this analysis
round was to have a model-independent result that is sensitive to a possible
deviation from the Standard Model. The cross section measurement has been
made within a phase-space defined to mimic the one used for the reconstruction
of Higgs boson decays in the four lepton final state. The fiducial cross section
measurements have been performed using a method based on a fit of the invariant
mass distribution. In addition to the previous published analysis with
\SI{13}{\tera\electronvolt} data, the differential cross section measurements
have been included and performed on seven variables of interest, which either
describe the Higgs kinematics ($p_{T}$, $y$, $\cos\theta^{∗}$, $m_{12}$ and
$m_{34}$) or are sensitive to the details of the Higgs boson production
($N_{\mathrm{jets}}$, $p_{T}^{lj}$, $m_{jj}$, $\Delta\phi_{jj}$, $\Delta\eta_{jj}$).
Also the first double-differential cross section measurements
($p_{T}$ versus $N_{\mathrm{jets}}$ and $m_{12}$ versus $m_{34}$) have been performed.
Given the sensitivity of $m_{12}$ versus $m_{34}$ kinematic plane to Beyond Standard
Model (BSM) couplings, limits on modified Higgs boson interactions within the
framework of pseudo-observables have been set. The results have been published
with \SI{36.1}{\per\femto\barn} of 13 \si{\tera\electronvolt} data as a CONF
note (\href{https://atlas.web.cern.ch/Atlas/GROUPS/PHYSICS/CONFNOTES/ATLAS-CONF-2017-032/}{ATLAS-CONF-2017-032})
and a paper (\href{https://link.springer.com/article/10.1007/JHEP10(2017)132}{JHEP 10 (2017) 132}).\\
A second round of the analysis has been published as a CONF note
(\href{https://atlas.web.cern.ch/Atlas/GROUPS/PHYSICS/CONFNOTES/ATLAS-CONF-2018-018/}{ATLAS-CONF-2018-018})
containing the results with \SI{80}{\per\femto\barn} of 13 \si{\tera\electronvolt}.\\
Currently I am involved in the next analysis round which aims to publish results
with full RUN2 data. I'm covering the role of analysis contact and main
analyser. I expect to improve the unfolding method employed (matrix inversion
instead of the bin-by-bin unfolding) and to consider some additional sensitive
variables. The change of the unfolding method requires a consistent
optimisation of the analysis framework: the efficiency matrix is built from MC
simulation, connecting the events kinematic at reconstruction level with the
one at particle level. Instead of making use of the truth matching tools
provided by ROOT, I performed a selection based on C++ pointers, storing just
the needed information on the shared memory. As a result, the analysis
framework gets faster by a factor 2 using a very low amount (around 500MB) of
memory when running on more than 15M events.
%
\end{itemize}
%
\textbf{Muon reconstruction and performance}\\
\vspace{-1.8em}
\begin{itemize}[labelwidth=0.05in,align=right,leftmargin=!,labelsep=0pt,
itemsep=0.0em]
\item[] Strong contribution to the measurement of the muon reconstruction
performance of the ATLAS detector. I evaluated the muon momentum resolution of
the ATLAS tracking system  as a function of the muon $p_{T}$ by using
data-driven techniques. The $Z$ boson decay into muons has been used. Since
the ATLAS detector is equipped with an Inner Detector (ID) and a Muon
Spectrometer (MS), combined muon tracks have been considered. The muon momentum
resolution has been obtained performing a ``global'' fit procedure in which
both the information of the ID and the MS are used. The input quantities of
the fit technique are the reconstructed di-muon invariant mass peak at the $Z$
pole, the difference between the independent momentum measurements of the ID
and the MS and external constraints on MS alignment and multiple scattering in
the ID and the MS. The final values of the muon momentum parameters have been
obtained using an iterative $\chi^{2}$ minimisation fit. The momentum resolution
corrections for physics analysis have been also derived and provided for physics
analyses making an official ATLAS analysis tool
(\texttt{MuonMomentumCorrections}). This work has been published in a paper
(\href{http://link.springer.com/article/10.1140\%2Fepjc\%2Fs10052-014-3130-x}{Eur. Phys. J. C74 (2014) 3130})
and two conference notes (\href{https://cds.cern.ch/record/1580207}{ATLAS-CONF-2013-088},
\href{https://atlas.web.cern.ch/Atlas/GROUPS/PHYSICS/CONFNOTES/ATLAS-CONF-2011-046/}{ATLAS-CONF-2011-046}).
\end{itemize}
%
\textbf{Monte Carlo generation and validation}\\
\vspace{-1.8em}
\begin{itemize}[labelwidth=0.05in,align=right,leftmargin=!,labelsep=0pt,
itemsep=0.0em]
\item[] \textsc{Higgs MC Production Manager}\\
I am currently covering the role of Higgs MC Production Manager since November
2016. I'm responsible for Monte Carlo production, validation and submission
of physics processes on behalf of the ATLAS Higgs group. Among all duties such
role implies, the most important one is to plan and effectively apply the best
strategy to simulate physics processes in agreement with the Higgs analysis
groups, Physics and Modelling group and Physics Coordination. Moreover, I am
responsible for JobOption preparation (python based) and validation, and any
other detail before actual submitting the request to the ATLAS Monte Carlo
production team. I also provide technical support on the Monte Carlo generator,
Parton Shower and PDF choice when needed.
%
\item[] \textsc{Responsible for Monte Carlo Simulation in the HZZ group}\\
In the context of the $H\to ZZ^{∗}\to 4l$ group, I have been responsible for 
the Monte Carlo production, between September 2012 and April 2014. Covering such
role, I configured, validated and planned several MC samples for both search
analysis (continuum ZZ background samples using the Powheg MC generator,
reducible $Z$+jets and $Z$+$bb$ samples using the Alpgen MC generator) and
properties measurements (Higgs signal samples for various spin and parity
hypotheses using the JHU MC generator).\\
Later on, I have been strongly involved in the preparation and optimisation of
MC samples production for the LHC RUNII. In particular I optimised the
production of the continuum ZZ background by using the Powheg on-the-fly
generation and validated the production machinery for the reducible $Z$+$bb$
background samples by using Sherpa MC generator.
\end{itemize}
%
\textbf{Computing and Software}\\
\vspace{-1.8em}
\begin{itemize}[labelwidth=0.05in,align=right,leftmargin=!,labelsep=0pt,
itemsep=0.0em]
\item[] I used the physics analysis framework developed in the context of the 
$H\to ZZ^{∗}\to 4\ell$ analysis to investigate a new parallel computing
framework based on the PROOF system and using the ATLAS Grid facilities.
I investigated the usage of Proof-on-Demand (PoD) to enable PROOF-based analysis
on ATLAS using the developed PoD/PanDA plug-in interface. Data has been
accessed using two different protocols: XRootD and file protocol. This work
has been presented in the 20th International Conference on Computing in High
Energy and Nuclear Physics (CHEP 2013, Amsterdam) and published on the Journal
of Physics (\href{http://iopscience.iop.org/1742-6596/513/3/032102}{J. Phys. 2014: Conf. Ser. 513 032102}).\\
In the context of the combined muon performance, I developed and maintained,
between 2011 and 2012, an official ATLAS analysis tool
(\texttt{MuonMomentumCorrections}), providing muon resolution, smearing
functions and corrections for physics analyses.\\
I also contributed to the development of the official ATLAS Monte Carlo
on-the-fly generation system, focusing on Alpgen and Prophecy MC generators.
\end{itemize}
%
}

\end{cvinterests}
