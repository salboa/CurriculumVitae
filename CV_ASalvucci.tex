%-----------------------------------------------------------------------------
%	PACKAGES AND OTHER DOCUMENT CONFIGURATIONS
%-----------------------------------------------------------------------------
\newcommand*{\TEXPATH}{classes/}
\documentclass[11pt, a4paper]{\TEXPATH awesome-cv}
\usepackage[english]{babel}
% A4 paper size by default, use 'letterpaper' for US letter

\geometry{left=2cm, top=1.5cm, right=2cm, bottom=2cm, footskip=.5cm}
% Configure page margins with geometry

%\fontdir[fonts/] % Specify the location of the included fonts

% Color for highlights
\colorlet{awesome}{awesome-emerald} % Default colors include: awesome-emerald, awesome-skyblue, awesome-red, awesome-pink, awesome-orange, awesome-nephritis,
%awesome-concrete, awesome-darknight
%\definecolor{awesome}{HTML}{CA63A8} % Uncomment if you would like to specify your own color

% Colors for text - uncomment and modify
%\definecolor{darktext}{HTML}{414141}
%\definecolor{text}{HTML}{414141}
%\definecolor{graytext}{HTML}{414141}
%\definecolor{lighttext}{HTML}{414141}

\renewcommand{\acvHeaderSocialSep}{\quad\textbar\quad}
% If you would like to change the social information separator from a pipe (|)
%to something else

\newcommand{\custItem}{\textbf{-}}
\newlength\widest
\settowidth\widest{ii}
\newcommand{\Achieve}{\underline{\color{graytext}{\textbf{Achievements}}}}
%--------------------------------------------------------------------------
%	PERSONAL INFORMATION
%	Comment any of the lines below if they are not required
%--------------------------------------------------------------------------

\photo[circle,edge,right]{picture}
\name{Antonio}{Salvucci}
\dob{Latina (IT) on July 13th, 1983}
\mobile{(+41) 076-23-66-338 , (+39) 327-76-99-320}
\email{s4lb04@gmail.com}
\github{salboa}
\googleScholar{https://scholar.google.it/citations?user=xhzjY6kAAAAJ\&hl=en}
\orcid{https://orcid.org/0000-0003-4876-2613}
\linkedin{antonio-salvucci}
\skype{s4lb04}
\address{Address: Rue du Jura, 14 - 1196 Gland (CH)}
\position{Physics Ph.D. | Software Engineer |  IT Manager}

\quote{
  An analytical software engineer and IT system manager, with a background
  and a PhD in particle physics. Experienced at developing cross-platform
  core and analysis sofware packages. Very comfortable at working in teams
  and independently in international enviroments. Experienced at providing
  IT support, either face to face or remotely, with a strong problem solving
  attitude. Naturally inclined to provide help and guidance and turn problems
  into opportunities.
  %challenges.
  %An analytical software developer and data scientist, with a background and a
  %Ph.D. in particle physics, experienced at working in teams and independently
  %in the world’s largest laboratory for high-energy physics research, CERN.
  %Experienced in managing projects and resources, turning problems into
  %opportunities, and extract meaning from data. On the lookout for exciting new
  %challenges.
}

\makecvfooter{\monthyeardate\today}{Antonio Salvucci~~~·~~~Résumé}{\thepage}
% Specify the letter footer with 3 arguments: (<left>, <center>, <right>),
%leave any of these blank if they are not needed
%-------------------------------------------------------------------------

%usufel definition
\DeclareSIUnit\pb{\pico\barn}
\DeclareSIUnit\fb{\femto\barn}
\DeclareSIUnit\cm{\centi\metre}
\DeclareSIUnit\mum{\micro\metre}
\DeclareSIUnit\mm{\milli\metre}
\DeclareSIUnit\mrad{\micro\radian}

% define conditions
\newif\ifAddAnnex
\newif\ifResearch
\newif\ifAddContacts
\newif\ifAddInterest

\Researchfalse
\AddAnnexfalse
\ifResearch
\AddContactsfalse
\AddInterestfalse
\else
\AddContactsfalse
\AddInteresttrue
\fi

\newcommand{\Genomsys}{
  \begin{minipage}{0.49\textwidth}
    \begin{itemize}[labelwidth=\the\widest,align=right,leftmargin=!,labelsep=1pt,noitemsep]
    \item[\custItem] Contribution to the conception, design and development of
      software tools and applications performing genomic data compression
      (MPEG-G standard)
    \item[\custItem] Genomic data compression validation and contribution to the
      implementation of genomic pipelines
    \item[\custItem] Benchmarking, validation and testing of software packages,
      both in development and in the release phase 
    \item[\custItem] Maintainer of Debian and Redhat based packages of the
      company software
    \end{itemize}
  \end{minipage}
  \hfill
  \begin{minipage}{0.49\textwidth}
    \begin{itemize}[labelwidth=\the\widest,align=right,leftmargin=!,labelsep=1pt,
        noitemsep]
    \item[\custItem] Implementation and administration of the company IT
      infrastructure both in terms of hardware and software systems
      (\textsc{\color{awesome}{Gnu/Linux servers}}, \textsc{\color{awesome}{Git}},
      \textsc{\color{awesome}{Vpn}}, \textsc{\color{awesome}{Jira}},
      \textsc{\color{awesome}{Confluence}}, \textsc{\color{awesome}{Bamboo}},
      \textsc{\color{awesome}{Jenkins}}, \textsc{\color{awesome}{Nextcloud}},
      \textsc{\color{awesome}{dhcp}}, \textsc{\color{awesome}{dns}},
      \textsc{\color{awesome}{Web services}})
    \item[\custItem] IT support to staff members in the configuration of
      desktop/laptop (\textsc{\color{awesome}{Windows}},
      \textsc{\color{awesome}{OSX}},  \textsc{\color{awesome}{Linux}}) and
      other IT devices
    \item[\custItem] Technical support to key partners and customers
    \item[\custItem] Implementation of storage systems and backup strategies,
      including disaster recovery plans
    \end{itemize}
  \end{minipage}
  \vspace{0.4em}
  
  \noindent \Achieve: planned and implemented the move of the existing IT
  infrastructure to new offices in February 2020. Configuration of 6 new servers
  for data storage and computing. Configuration of the ticket managament system Jira,
  the collaboration platform Confluence, the continuous integration platforms Bamboo
  and Jenkins, the Virtual Private Network (VPN) and the Domain Name Service (DNS).\\
  Developed new features in all the 4 releases of the genomic data compression software
  packages. Carried out tests and validations of the genomic data compression to
  verify its lossless. Investigated the usage of the genomic compressed data in genomic
  pipeline, verifying the analysis results to be consistent with those obtained with
  uncompressed data.
}

\newcommand{\CHUK}{
  \begin{minipage}{0.49\textwidth}
    \begin{itemize}[labelwidth=\the\widest,align=right,leftmargin=!,labelsep=1pt,
    noitemsep]
    \item[\custItem] Main software developer, team leader and paper author for
      two analysis projects (search for new physics signals and Higgs boson
      properties measurements)
    \item[\custItem] Setup and management of student's desktop/laptop (Linux,
      OSX and Windows) and related troubleshooting
    \end{itemize}
  \end{minipage}
  \hfill
  \begin{minipage}{0.49\textwidth}
    \begin{itemize}[labelwidth=\the\widest,align=right,leftmargin=!,labelsep=1pt,noitemsep]
    \item[\custItem] Supervision of Master and PhD students
    \item[\custItem] Responsible for data simulation, modeling and validation
      using Monte Carlo (MC) generators
    \item[\custItem] \emph{MC production manager}: coordination and support of
      MC production for all Higgs analyses in ATLAS (extensive use of JIRA)
    \end{itemize}
  \end{minipage}
  \vspace{0.4em}
  
  \noindent \Achieve: 3 papers in peer-reviewed journals
  (\scriptsize{links: \href{https://journals.aps.org/prd/abstract/10.1103/PhysRevD.98.092008}{\textcolor{awesome-emerald}{a}},
  \href{https://link.springer.com/article/10.1007/JHEP10(2017)132}{\textcolor{awesome-emerald}{b}},
  \href{https://link.springer.com/article/10.1140/epjc/s10052-016-4385-1}{\textcolor{awesome-emerald}{c}}}); 2 talks and 1 poster at international conferences.
}
\newcommand{\JuniorScientist}{
  \begin{minipage}{0.49\textwidth}
    \begin{itemize}[labelwidth=\the\widest,align=right,leftmargin=!,labelsep=1pt,noitemsep]
    \item[\custItem] Main software developer and main analyser for the Higgs
      boson properties measurements
    \end{itemize}
  \end{minipage}
  \begin{minipage}{0.49\textwidth}
    \begin{itemize}[labelwidth=\the\widest,align=right,leftmargin=!,labelsep=1pt,noitemsep]
    \item[\custItem] Responsible for data simulation, modeling and validation
    \item[\custItem] Supervision of Master students
    \end{itemize}
  \end{minipage}
  \vspace{0.4em}
  
  \noindent \Achieve: 4 papers in peer-reviewed journals
  (\scriptsize{links:
    \href{https://link.springer.com/article/10.1140/epjc/s10052-015-3685-1}{\textcolor{awesome-emerald}{d}},
  \href{https://journals.aps.org/prd/abstract/10.1103/PhysRevD.91.012006}{\textcolor{awesome-emerald}{e}},
  \href{http://www.sciencedirect.com/science/article/pii/S0370269314007126}{\textcolor{awesome-emerald}{f}},
  \href{http://journals.aps.org/prd/abstract/10.1103/PhysRevD.90.052004}{\textcolor{awesome-emerald}{g}}}); 1 talk at international conference.
}
\newcommand{\Phd}{
  \begin{minipage}{0.49\textwidth}
    \begin{itemize}[labelwidth=\the\widest,align=right,leftmargin=!,labelsep=1pt,
        noitemsep]
    \item[\custItem] Main software developer and main analyser for two analysis teams
      (Higgs boson discovery/properties, detector performance)
    \item[\custItem] Responsible for MC production and validation
    \item[\custItem] Assistant for physics laboratories and C++ programming
      course
    \end{itemize}
  \end{minipage}
  \hfill
  \begin{minipage}{0.49\textwidth}
    \begin{itemize}[labelwidth=\the\widest,align=right,leftmargin=!,labelsep=1pt,noitemsep]
    \item[\custItem] Tester of a new parallel computing approach using
      Proof-on-Demand on ATLAS computing clusters
    \item[\custItem] Maintainer of Debian-based packages for the
      \textsc{\color{awesome}{ROOT}} sofware on the CERN Debian/Ubuntu
      repository
    \end{itemize}
  \end{minipage}
  \vspace{0.4em}
  
  \noindent \Achieve: parallel computing and machine learning
  techniques; contribution to 9 papers in peer-reviewed journals
  (\scriptsize{links: 
  \href{http://link.springer.com/article/10.1140\%2Fepjc\%2Fs10052-014-3130-x}{\textcolor{awesome-emerald}{h}},
  \href{link.springer.com/article/10.1140/epjc/s10052-014-3034-9}{\textcolor{awesome-emerald}{i}},
  %\href{http://www.sciencedirect.com/science/article/pii/S0370269313006369}{\textcolor{awesome-emerald}{j}},
  \href{http://www.sciencedirect.com/science/article/pii/S0370269313006527}{\textcolor{awesome-emerald}{j}},
  \href{http://www.sciencedirect.com/science/article/pii/S037026931200857X}{\textcolor{awesome-emerald}{k}},
  \href{http://journals.aps.org/prd/abstract/10.1103/PhysRevD.86.032003}{\textcolor{awesome-emerald}{l}},
  \href{https://www.epj-conferences.org/articles/epjconf/abs/2012/10/epjconf_hcp2012_12039/epjconf_hcp2012_12039.html}{\textcolor{awesome-emerald}{m}},
  \href{https://www.sif.it/riviste/sif/ncc/econtents/2013/036/06/article/12}{\textcolor{awesome-emerald}{n}},
  \href{http://iopscience.iop.org/1742-6596/513/3/032102}{\textcolor{awesome-emerald}{o}}});
  4 talks and 2 posters at international conferences.
}
\newcommand{\CERN}{Software developer for two analysis teams (detector
  performance and search for new physics signals)}
\newcommand{\Master}{
  \begin{minipage}{0.49\textwidth}
  \begin{itemize}[labelwidth=\the\widest,align=right,leftmargin=!,labelsep=1pt,noitemsep]
  \item[\custItem] Software developer and statistical treatment for limit setting
  \item[\custItem] Detector performance and MC production and validation
  \end{itemize}
  \end{minipage}
  \hfill
  \begin{minipage}{0.49\textwidth}
    \begin{itemize}[labelwidth=\the\widest,align=right,leftmargin=!,labelsep=1pt,noitemsep]
    \item[\custItem] Setup of a small local computing cluster and tester of the
      ATLAS Tier2 computing cluster in Frascati
    \end{itemize}
  \end{minipage}
}
\newcommand{\PhdThesis}{\textbf{Thesis}: The Higgs boson in the
  $H\to ZZ^{∗}\to 4\ell$ decay channel with the ATLAS detector at the LHC}
\newcommand{\MsThesis}{\textbf{Thesis}: Search for new gauge boson in
  collisions $pp\to Z^{\prime}+X\to\mu^{+}\mu^{-}+X$ with the ATLAS detector at
  the LHC}
\newcommand{\BsThesis}{\textbf{Thesis}: A new experimental limit for the
  Pauli Exclusion Principle for electrons}
\newcommand{\skipV}{\vspace{0.0em}}
\ifResearch
\renewcommand{\CurStatDet}{Software developer, analysis contact and publications
  editor in the context of the search for new possible heavy neutral
  lepton-flavor violating ($LFV$) dilepton final states with the ATLAS
  experiment at the LHC. Software developer, analysis contact and internal
  documentation editor for the Higgs boson fiducial and differential cross
  section measurements in the $H\to ZZ^{∗}\to 4\ell$ decay channel. Responsible
  for data simulation and modeling using Monte Carlo (MC) generators: I
  provided user support for the production of official MC sample with the ATLAS
  Higgs group (role of \emph{Higgs MC production manager}). Supervisor of
  Master and PhD students.}
\renewcommand{\JuniorScientist}{Software developer and main analyser for Higgs
  boson properties measurements in the $H\to ZZ^{∗}\to 4\ell$ decay channel with
  the ATLAS experiment at the LHC. I lead my analysis team for the analysis
  preparation for the LHC RUNII data taking: main activities include Monte
  Carlo generation and validation at \SI{13}{\tera\electronvolt}, selection
  optimisation and porting of the analysis framework to the new ATLAS RUNII
  data format (xAODs).}
\renewcommand{\Phd}{Main analyser and framework developer in the context of the
  Higgs boson search and properties measurements (couplings, spin) in the
  $H\to ZZ^{∗}\to 4\ell$ decay channel with the ATLAS experiment at the LHC.
  I optimised the selection criteria to reject background events and maximize
  the signal significance by improving the kinematic selection and applying
  machine learning techniques. I evaluated the ATLAS tracking system performance
  and measured the muon momentum resolution of the ATLAS detector. I was
  responsible for the Monte Carlo generation and validation in the $H\to ZZ^{∗}$
  group. I also investigated the usage of Proof-on-Demand (PoD) to enable
  PROOF-based analyses on ATLAS computing clusters. I was assistant for
  physics laboratories and C++ programming course.}
\renewcommand{\CERN}{I evaluated the muon momentum resolution of the ATLAS
  detector at the LHC and searched for a new heavy neutral resonance
  ($Z^{\prime}$) decaying in a muon pair.}
\renewcommand{\Master}{Software developer for the evaluation of the discovery
  potential of the ATLAS experiment for a possible new heavy neutral resonance
  ($Z^{\prime}$) decaying in a muon pair: I set limits on the production cross
  section and studied decay properties using MC simulation and toy-experiments.
  I evaluated the performance of the ATLAS tracking system, such as momentum
  resolution and tracking and trigger efficiencies. I also measured the impact
  of detector mis-alignments on physics analyses and produced and validated
  Monte Carlo samples.}
\renewcommand{\PhdThesis}{\textbf{Thesis}: The Higgs boson in the
  $H\to ZZ^{∗}\to 4\ell$ decay channel with the ATLAS detector at the LHC}
\renewcommand{\MsThesis}{\textbf{Thesis}: Search for new gauge boson in
  collisions $pp\to Z^{\prime}+X\to\mu^{+}\mu^{-}+X$ with the ATLAS detector at
  the LHC}
\renewcommand{\BsThesis}{\textbf{Thesis}: A new experimental limit for the
  Pauli Exclusion Principle for electrons}
\renewcommand{\skipV}{\vspace{0.0em}}
\fi

\begin{document}

\makecvheader % Print the header
%--------------------------------------------------------------------------
% CV/RESUME CONTENT
% Each section is imported separately, open each file in turn to modify content
%--------------------------------------------------------------------------

%------------------------------------------------------------------------
%	SECTION TITLE
%------------------------------------------------------------------------
\vspace{-2.5em}

\cvsection{{Proven Capabilities}}

%------------------------------------------------------------------------
%	SECTION CONTENT
%------------------------------------------------------------------------

\begin{cvskills}

\cvskill
{Software Development and Data Analysis}%title
{
\begin{minipage}{0.49\textwidth}
\begin{itemize}[labelwidth=\the\widest,align=right,leftmargin=!,labelsep=1pt,noitemsep]
\item[\custItem] Development and maintenance of core, statistical and analysis
algorithms (languages \textsc{\color{awesome}{C++}} and
\textsc{\color{awesome}{Python}})
\item[\custItem] Statistical analysis of massive volumes of high energy physics
data also via machine learning techniques (\textsc{\color{awesome}{TMVA}},
\textsc{\color{awesome}{BDT}}, \textsc{\color{awesome}{TensorFlow}})
\item[\custItem] Scripting languages (\textsc{\color{awesome}{bash}},
\textsc{\color{awesome}{sh}}, \textsc{\color{awesome}{ksh}},
\textsc{\color{awesome}{csh}})
\end{itemize}
\end{minipage}
\hfill
\begin{minipage}{0.49\textwidth}
\begin{itemize}[labelwidth=\the\widest,align=right,leftmargin=!,labelsep=1pt,noitemsep]
\item[\custItem] Experience in improving software performance (time and memory
consumption by using \textsc{\color{awesome}{top/htop}} and
\textsc{\color{awesome}{Valgrind}})
\item[\custItem] Design/optimisation of algorithms for computing
clusters/clouds
\item[\custItem] High energy physics simulation
(\textsc{\color{awesome}{Powheg}}, \textsc{\color{awesome}{Pythia}},
\textsc{\color{awesome}{Madgraph}}) and data analysis
(\textsc{\color{awesome}{ROOT}}) tools
\end{itemize}
\end{minipage}
}
%------------------------------------------------

\cvskill
{Information Technology}
{
\begin{minipage}{0.48\textwidth}
\begin{itemize}[labelwidth=\the\widest,align=right,leftmargin=!,labelsep=1pt,noitemsep]
\item[\custItem] Installation, setup and management of operating systems
(\textsc{\color{awesome}{Linux/Unix}}, \textsc{\color{awesome}{Windows}},
\textsc{\color{awesome}{MacOS}}) and services (\textsc{\color{awesome}{ssh}},
\textsc{\color{awesome}{ftp}},\textsc{\color{awesome}{nfs}}, etc).
\item[\custItem] Ability in writing rules to create pre-compiled packages for
Debian/Ubuntu (maintainer between 2009 and 2012 of \textsc{\color{awesome}{ROOT}}
packages for the CERN repository)
\item[\custItem] Version control systems (\textsc{\color{awesome}{Git}},
\textsc{\color{awesome}{GitLab}}, \textsc{\color{awesome}{svn}},
\textsc{\color{awesome}{cvs}}), project and issue tracking software
(\textsc{\color{awesome}{JIRA}})
\item[\custItem] Web applications (\textsc{\color{awesome}{Angular}})
\end{itemize}
\end{minipage}
\hfill
\begin{minipage}{0.48\textwidth}
\begin{itemize}[labelwidth=\the\widest,align=right,leftmargin=!,labelsep=1pt,noitemsep]
\item[\custItem] Document management systems (\textsc{\color{awesome}{twiki}},
\textsc{\color{awesome}{wiki}})
\item[\custItem] Application software
(\textsc{\color{awesome}{Microsoft Office}},
\textsc{\color{awesome}{Google Doc Suite}})
\item[\custItem] Numerical computing applications
(\textsc{\color{awesome}{Matlab}}, \textsc{\color{awesome}{Mathematica}})
\item[\custItem] Writing tools ({\color{awesome}{\LaTeX}},
{\color{awesome}{\TeX}})
\item[\custItem] Video (\textsc{\color{awesome}{Adobe Premiere}}) and
image (\textsc{\color{awesome}{Adobe Photoshop}},
\textsc{\color{awesome}{Gimp}}) editing applications
\item[\custItem] Data query and manipulation(\textsc{\color{awesome}{sql}}, 
\textsc{\color{awesome}{json}},\textsc{\color{awesome}{xml}},
\textsc{\color{awesome}{csv}})
\end{itemize}
\end{minipage}
}
%------------------------------------------------

\cvskill
{Teamwork, Project Management and Scientific Communication} % Title
{
\begin{minipage}{0.49\textwidth}
\begin{itemize}[labelwidth=\the\widest,align=right,leftmargin=!,labelsep=1pt,noitemsep]
\item[\custItem] Effective teamwork within analysis teams of 5-40 people
\item[\custItem] Coordination of teams of 5-10 members and research project
management
\item[\custItem] Supervision of Master and PhD students
%meetings organisation, tasks assignment according to team-member's attitude
%and skills, setting priorities to achieve
%agreed goals within defined time-frames
%\item[\custItem] Efficient at taking actions that require
%logic, mental order and maintenance
\end{itemize}
\end{minipage}
\hfill
\begin{minipage}{0.49\textwidth}
\begin{itemize}[labelwidth=\the\widest,align=right,leftmargin=!,labelsep=1pt,noitemsep]
\item[\custItem] Author of scientific papers published in peer-reviewed
journals
\item[\custItem] Vivid interpersonal and public speaking skills
\item[\custItem] University assistant in software development and physics
laboratories
\end{itemize}
\end{minipage}
\begin{itemize}[labelwidth=\the\widest,align=right,leftmargin=!,labelsep=1pt,noitemsep]
\item[] \textsc{\color{awesome}Languages:} English (C1), French (A2),
Italian (mother tongue)
\end{itemize}
}
%------------------------------------------------
%
%\cvskill
%{Scientific Communication} % Title
%{
%\begin{minipage}{0.49\textwidth}
%\begin{itemize}[labelwidth=\the\widest,align=right,leftmargin=!,labelsep=1pt,noitemsep]
%\item[\custItem] Scientific author for peer-reviewed journals
%\item[\custItem] Vivid interpersonal and public speaking skills
%\end{itemize}
%\end{minipage}
%\begin{minipage}{0.49\textwidth}
%\begin{itemize}[labelwidth=\the\widest,align=right,leftmargin=!,labelsep=1pt,noitemsep]
% \item[\custItem] Students' supervision
%\item[\custItem] Assistant at the Radboud University
%\end{itemize}
%\end{minipage}
%\begin{itemize}[labelwidth=\the\widest,align=right,leftmargin=!,labelsep=1pt,noitemsep]
%\item[] \textsc{\color{awesome}Languages:} English (C1), French (A1),
%Italian (mother tongue)
%\end{itemize}
%}

\end{cvskills}

%------------------------------------------------------------------------
%	SECTION TITLE
%------------------------------------------------------------------------
\cvsection{{Working Experience}}

%------------------------------------------------------------------------
%	SECTION CONTENT
%------------------------------------------------------------------------
\begin{cventries}
\cventry
{Genomsys} % Organization
{IT Manager and Software Engineer} % Job title
{Dec. 2019 - Present} % Date(s)
{Lausanne (CH)} % Location
{\Genomsys}

\cventry
{The Chinese University of Hong Kong / CERN} % Organization
{Post-Doc Research Fellow} % Job title
{Jun. 2015 - Sep. 2019} % Date(s)
{Hong Kong (HK) / Geneva (CH)}
{\CHUK}

\cventry
{Radboud University Nijmegen / CERN} % Organization
{Junior Scientist} % Job title
{Apr. 2014 - Jan. 2015} % Date(s)
{Nijmegen (NL) / Geneva (CH)} % Location
{\JuniorScientist}
%------------------------------------------------

\cventry
{Radboud Univesity Nijmegen / CERN} % Organization
{Ph.D. Researcher} % Job title
{Feb. 2011  - Mar. 2014} % Date(s)
{Nijmegen (NL) / Geneva (CH)} % Location
{\Phd}
%------------------------------------------------

\cventry
{CERN} % Organization
{Young Associate Researcher} % Job title
{Nov. 2010 - Jan. 2011} % Date(s)
{Geneva (CH)} % Location
{\CERN}
%------------------------------------------------

\cventry
{INFN-LNF} % Organization
{Scholarship for Undergraduates} % Job title
{Feb. 2009 - Feb. 2010} % Date(s)
{Frascati (IT)} % Location
{\Master}
%------------------------------------------------

\cventry
{Associazione Italiana Arbitri (AIA)} % Organization
{Football Referee} % Job title
{2000 - 2005} % Date(s)
{Latina (IT)} % Location
{Refereed more than 100 matches in the Italian youth football league}

\end{cventries}

\ifAddContacts
%----------------------------------------------------------------------
%	SECTION TITLE
%----------------------------------------------------------------------
\cvsection{{Working References}}

%----------------------------------------------------------------------
%	SECTION CONTENT
%----------------------------------------------------------------------
\begin{cventries}

%------------------------------------------------
\cventry
{Radboud University Nijmegen} % Organization
{Nicolo de Groot {\normalfont\bodyfont\footnotesize (Ph.D. promoter)}} % Name
{N.deGroot@science.ru.nl} % mail
{Nijmegen, The Netherlands}
{} %position
\vspace{-0.9em}

\cventry
{Radboud University Nijmegen} % Organization
{Frank Filthaut {\normalfont\bodyfont\footnotesize (Ph.D. supervisor)}} % Name
{f.filthaut@science.ru.nl} % mail
{Nijmegen, The Netherlands}
{}
\vspace{-0.9em}

\cventry
{Lawrence Berkeley National Laboratory and CERN} % Organization
{Fabio Cerutti {\normalfont\bodyfont\footnotesize (Main senior collaborator at CERN)}} % Name
{fabio.cerutti@cern.ch} % mail
{Meyrin, Switzerland}
{}
\vspace{-1.1em}

\end{cventries}

\fi
%-----------------------------------------------------------------------
%	SECTION TITLE
%-----------------------------------------------------------------------
\cvsection{{Education}}

%-----------------------------------------------------------------------
%	SECTION CONTENT
%-----------------------------------------------------------------------

\begin{cveducations}

%------------------------------------------------
\cveducation
{Radboud University Nijmegen} % Institution
{Ph.D. in High Energy Physics (Hons)} % Degree
{2011-2014} % Date(s)
{Nijmegen, The Netherlands} % Location
{\PhdThesis} %ThesisTitle
%------------------------------------------------
\skipV

\cveducation
{Universit\'a degli Studi di Roma ``Tor Vergata''} % Institution
{M.Sc. in Particle Physics} % Degree
{2006-2010} % Date(s)
{Rome, Italy} % Location
{\MsThesis} %ThesisTitle
%------------------------------------------------
\skipV

\cveducation
{Universit\'a degli Studi di Roma ``Tor Vergata''} % Institution
{B.Sc. in Physics} % Degree
{2002-2006} % Date(s)
{Rome, Italy} % Location
{\BsThesis} %ThesisTitle
%------------------------------------------------
\skipV

\end{cveducations}

%-----------------------------------------------------------------------
%       SECTION TITLE
%-----------------------------------------------------------------------
\cvsection{{Certificates}}

%-----------------------------------------------------------------------
%       SECTION CONTENT
%-----------------------------------------------------------------------

\begin{cvcertificates}

%------------------------------------------------
\cvcertificate
{Coursera Course Certificate} % Source
{Divide and Conquer, Sorting and Searching, and Randomized Algorithms} % Title
{Nov 2019} % Date(s)
{\href{https://www.coursera.org/account/accomplishments/verify/ZSWGXT5RRQDG}{link}}
%------------------------------------------------
%\skipV
%------------------------------------------------
\cvcertificate
{Coursera Course Certificate} % Source
{System Administration and IT Infrastructure Services} % Title
{Jun 2022} % Date(s)
{\href{https://coursera.org/verify/FN7ULWH2B6HG}{link}}
%------------------------------------------------

\end{cvcertificates}

\ifResearch
%----------------------------------------------------------------------
%	SECTION TITLE
%----------------------------------------------------------------------
\cvsection{{Talks at conferences and workshops}}

%----------------------------------------------------------------------
%	SECTION CONTENT
%----------------------------------------------------------------------
\begin{cvtalks}

%------------------------------------------------
\cvtalk
{Cross Sections measurements: review of the current analysis with \SI{36.1}{\per\femto\barn} of data at $\sqrt{s}=13$ \si{\tera\electronvolt}} %title
{ATLAS Higgs to ZZ Workshop} %ConfName
{Oxford, UK} %place
{10-13 April 2018} %date

\cvtalk
{Lepton number violation searches at the LHC} %TalkTitle
{BLV 2017} %ConfName
{Cleveland, Ohio (USA)} %place
{15-18 May 2017} %date

\cvtalk
{Signatures and techniques: MVA and MEM for Higgs properties measurements} % TalkTitle
{ATLAS Higgs Workshop} %ConfName
{Rome, Italy} %place
{14-18 April 2014} %date

\cvtalk
{Higgs to $ZZ^{*}$ to $4\ell$: multidimensional analysis} %TalkTitle
{ATLAS Higgs to ZZ Workshop} %ConfName
{Rome, Italy} %place
{15-19 April 2013} %date

\cvtalk
{Recent Higgs $\to ZZ^{*}\to 4\ell$ results: latest results form the search for
the Standard Model Higgs Boson with the ATLAS experiment at the LHC} %TalkTitle
{Les Recontres de Physique de la Vall\'ee d'Aoste} %ConfName
{La Thuile, Italy} %place
{February 24th - March 2nd 2013} %date

\cvtalk
{Observation of a Higgs-like particle in the search for the Standard Model Higgs
Boson in the $H\to ZZ^{*}\to 4\ell$ with the ATLAS detector at the LHC} %TalkTitle
{Physics@FOM 2013} %ConfName
{Veldhoven, The Netherlands} %place
{22-23 January 2013} %date

\cvtalk
{Measurement of muon momentum resolution using the ATLAS detector} %TalkTitle
{NNV Dutch Physical Society Meeting} %ConfName
{Lunteren, The Netherlands} %place
{November 4th, 2011} %date

\end{cvtalks}


%----------------------------------------------------------------------
%	SECTION TITLE
%----------------------------------------------------------------------
\cvsection{{Posters at conferences}}

\cvtalk
{Measurement of the Higgs boson fiducial and differential cross sections in the
$4\ell$ decay channel at \SI{13}{\tera\electronvolt} with the ATLAS detector} %TalkTitle
{EPS-HEP 2017} %ConfName
{Venice, Italy} %place
{5-12 July 2017} %date

\cvtalk
{The new Higgs particle in the $H\to ZZ^{∗}\to 4\ell$ searches with the ATLAS
detector} %TalkTitle
{EPS-HEP 2013} %ConfName
{Stockhom, Sweden} %place
{18-24 July 2013} %date

\cvtalk
{Measurement of muon momentum resolution of the ATLAS detector} %TalkTitle
{HCP 2011} %ConfName
{Paris, France} %place
{November 2011} %date

%-----------------------------------------------------------------------
%	SECTION TITLE
%-----------------------------------------------------------------------
\cvsection{{Responsibilities in ATLAS}}

%-----------------------------------------------------------------------
%	SECTION CONTENT
%-----------------------------------------------------------------------
\begin{cvinterests}

%------------------------------------------------

\cvinterest{
%
\textbf{Higgs and Physics/Modelling group}\\
\vspace{-2em}
\begin{itemize}[labelwidth=0.05in,align=right,leftmargin=!,labelsep=0pt,
itemsep=0.0em]
\item[] \textsc{\color{awesome}{Higgs MC production manager}} (since Nov. 2016).
Responsible for Monte Carlo production, validation and submission on behalf of
the Higgs group (obtaining Class 3 OTP credits).
\end{itemize}
%
\textbf{Higgs HZZ group}\\
\vspace{-2em}
\begin{itemize}[labelwidth=0.05in,align=right,leftmargin=!,labelsep=0pt,
itemsep=0.0em]
\item[] \textsc{\color{awesome}{analysis contact}} (since Jul. 2018) for the
measurement of the fiducial and differential cross sections measurements;
\item[] \textsc{\color{awesome}{editor}} (Oct. 2017 - Jun. 2018) of the
supporting documentation for CONF note (\href{https://atlas.web.cern.ch/Atlas/GROUPS/PHYSICS/CONFNOTES/ATLAS-CONF-2018-018/}{ATLAS-CONF-2018-018})
showing the results of the Higgs boson properties measurements (fiducial and
differential cross sections and couplings) in the context of the $4\ell$ decay
channel using \SI{80}{\per\femto\barn} of $pp$ collision data at $\sqrt{s}=13$
\si{\tera\electronvolt};
\item[] \textsc{\color{awesome}{editor}} (Oct. 2016 - May 2017) of the
supporting documentation for the paper (\href{https://link.springer.com/article/10.1007/JHEP10(2017)132}{JHEP 10 (2017) 132})
and CONF note (\href{http://cdsweb.cern.ch/record/2265796}{ATLAS-CONF-2017-032})
on fiducial and differential cross sections measurements published using
\SI{36.1}{\per\femto\barn} of $pp$ collision data at
$\sqrt{s}=13$ \si{\tera\electronvolt};
\item[] \textsc{\color{awesome}{responsible}} for Monte Carlo validation (signal
and background samples) and submission between September 2012 and April 2014.
\end{itemize}
%
\textbf{Lepton+X/Exotics group}\\
\vspace{-2em}
\begin{itemize}[labelwidth=0.05in,align=right,leftmargin=!,labelsep=0pt,
itemsep=0.0em]
\item[] \textsc{\color{awesome}{analysis contact}} (since Sep. 2016) for
the search for lepton-flavor violating high-mass dilepton final states and 
\textsc{\color{awesome}{editor}} of the paper 
(\href{https://journals.aps.org/prd/abstract/10.1103/PhysRevD.98.092008}{Phys. Rev. D 98, 092008}) and corresponding supporting documentation published using
\SI{36.1}{\per\femto\barn} of $pp$ collision data at
$\sqrt{s}=13$ \si{\tera\electronvolt}.
\item[] \textsc{\color{awesome}{editor}} (Dec. 2015 - Aug. 2016) of the
paper (\href{https://link.springer.com/article/10.1140/epjc/s10052-016-4385-1}{Eur. Phys. J. C 76 (2016) 541}),
CONF note (\href{https://cds.cern.ch/record/2114844}{ATLAS-CONF-2015-072}) and
supporting documentation in the previous round of the analysis (search for
lepton-flavor violating high-mass dilepton final states) published using
\SI{3.2}{\per\femto\barn} of $pp$ collision data at
$\sqrt{s}=13$ \si{\tera\electronvolt}.
\item[] \textsc{\color{awesome}{editorial board member}} for the search for
lepton flavor violation of $Z$ boson decaying to electron and muon.
\end{itemize}
%
\textbf{Muon Combined Performance group}\\
\vspace{-2em}
\begin{itemize}[labelwidth=0.05in,align=right,leftmargin=!,labelsep=0pt,
itemsep=0.0em]
\item[] \textsc{\color{awesome}{developer and maintainer}} (Feb 2011 - Oct 2012)
of the \texttt{MuonMomentumCorrection} software package, which provides muon
resolution, smearing functions and corrections for physics analyses (obtained
Class 3 OTP credits for muon reconstruction).
\end{itemize}
%
%\textbf{Monte Carlo Simulation/Generator Software}:\\
%\vspace{-2em}
%\begin{itemize}[labelwidth=0.05in,align=right,leftmargin=!,labelsep=0pt,
%itemsep=0.0em]
%\item[] obtaining Class 3 OTP credits for Monte Carlo simulation and generation
%software (i.e. Powheg and Alpgen on-the-fly generation improvement);
%\end{itemize}
%%
}

\end{cvinterests}

\clearpage
%-----------------------------------------------------------------------
%	SECTION TITLE
%-----------------------------------------------------------------------
\cvsection{{Relevant Publications}}

%-----------------------------------------------------------------------
%	SECTION CONTENT
%-----------------------------------------------------------------------
\begin{cvinterests}

%------------------------------------------------

\cvinterest{
\vspace{-1.8em}
\begin{itemize}[labelwidth=0.05in,align=right,leftmargin=!,labelsep=0pt,
itemsep=0.0em]
%
\item[] \href{https://journals.aps.org/prd/abstract/10.1103/PhysRevD.98.092008}{\textbf{Phys. Rev. D 98, 092008 }},
ATLAS Collaboration, Search for new phenomena in different-flavour,
high-mass dilepton final states using \SI{36.1}{\per\femto\barn} of data from
$pp$ collisions at $\sqrt{s}=13$ \si{\tera\electronvolt} with the ATLAS
detector, (To be Submitted to PRD - Editor)
%
\item[] \href{https://link.springer.com/article/10.1007/JHEP10(2017)132}{\textbf{JHEP 10 (2017) 132}}, 
ATLAS Collaboration, Measurement of inclusive and differential cross sections
in the $H\to ZZ^{*}\to 4\ell$ decay channel in $pp$ collisions at $\sqrt{s}=13$
\si{\tera\electronvolt} with the ATLAS detector (2017)
%
\item[] \href{https://link.springer.com/article/10.1140/epjc/s10052-016-4385-1}{\textbf{Eur. Phys. J. C 76 (2016) 541}},
ATLAS Collaboration, Search for new phenomena in different-flavour high-mass
dilepton final states in pp collisions at $\sqrt{s}=13$ \si{\tera\electronvolt}
with the ATLAS detector (2016 - Editor)
%
\item[] \href{https://link.springer.com/article/10.1140/epjc/s10052-015-3685-1}{\textbf{Eur. Phys. J. C75 (2015) 476}},
ATLAS Collaboration, Study of the spin and parity of the Higgs boson in diboson
decays with the ATLAS detector (2015)
%
\item[] \href{https://journals.aps.org/prd/abstract/10.1103/PhysRevD.91.012006}{\textbf{Phys. Rev. D 91, 012006 (2015)}},
ATLAS Collaboration, Measurements of the Higgs boson production and couplings
in the four lepton decay channel with the ATLAS detector using
\SI{25}{\per\femto\barn} of proton-proton collision data (2015)
%
\item[] \href{http://www.sciencedirect.com/science/article/pii/S0370269314007126}{\textbf{Phys. Lett. B 738 (2014) 234-253}},
ATLAS Collaboration, Fiducial and differential cross sections of Higgs boson
production measured in the four-lepton decay channel in $pp$ collisions at a
centre-of-mass energy of \SI{8}{\tera\electronvolt} with the ATLAS detector
(2014)
%
\item[] \href{http://journals.aps.org/prd/abstract/10.1103/PhysRevD.90.052004}{\textbf{Phys. Rev. D 90, 052004 (2014)}},
ATLAS Collaboration, Measurement of the Higgs boson mass from the
$H\to\gamma\gamma$ and $H\to ZZ^{*}\to 4l$ channels with the ATLAS detector
using \SI{25}{\per\femto\barn} of $pp$ collision data (2014)
%
\item[] \href{http://link.springer.com/article/10.1140\%2Fepjc\%2Fs10052-014-3130-x}{\textbf{Eur. Phys. J. C74 (2014) 3130}},
ATLAS Collaboration, Measurement of the muon reconstruction performance of
the ATLAS detector using 2011 and 2012 LHC proton-proton collision data (2014)
%
\item[] \href{link.springer.com/article/10.1140/epjc/s10052-014-3034-9}{\textbf{Eur. Phys. J. C (2014) 74:3034}}, 
ATLAS Collaboration, Muon Reconstruction Efficiency and Momentum Resolution of
the ATLAS Experiment in Proton-Proton Collisions at $\sqrt{s}=7$
\si{\tera\electronvolt} in 2010 (2014)
%
\item[] \href{http://www.sciencedirect.com/science/article/pii/S0370269313006369}{\textbf{Phys. Lett. B 726 (2013), pp. 88-119}},
ATLAS Collaboration, Measurements of Higgs boson production and couplings in
diboson final states with the ATLAS detector at the LHC (2013)
%
\item[]  \href{http://www.sciencedirect.com/science/article/pii/S0370269313006527}{\textbf{Phys. Lett. B 726 (2013), pp. 120-144}},
ATLAS Collaboration, Evidence for the spin-0 nature of the Higgs boson using
ATLAS data (2013)
%
\item[] \href{http://www.sciencedirect.com/science/article/pii/S037026931200857X}{\textbf{Phys. Lett. B 716 (2012) 1-29}},
ATLAS Collaboration, Observation of a new particle in the search for the
Standard Model Higgs boson with the ATLAS detector at the LHC (2012)
%
\item[] \href{http://journals.aps.org/prd/abstract/10.1103/PhysRevD.86.032003}{\textbf{Phys. Rev. D86 (2012) 032003}},
ATLAS Collaboration, Combined search for the Standard Model Higgs boson in $pp$
collisions at $\sqrt{s}=7$ \si{\tera\electronvolt} with the ATLAS detector
(2012)
%
\item[] \href{http://www.sciencedirect.com/science/article/pii/S0370269312002560}{\textbf{Phys.Lett. B710 (2012) 383-402}},
ATLAS Collaboration, Search for the Standard Model Higgs boson in the decay
channel $H\to ZZ^{*}\to 4\ell$ with \SI{4.8}{\per\femto\barn} of $pp$ collision
data at $\sqrt{s}=7$ \si{\tera\electronvolt} with ATLAS (2012)
%
\item[] \href{http://journals.aps.org/prd/abstract/10.1103/PhysRevD.85.072004}{\textbf{Phys. Rev. D 85, 072004 (2012)}},
ATLAS Collaboration, Measurement of the inclusive $W^{\pm}$ and $Z/\gamma^{*}$
cross sections in the $e$ and $\mu$ decay channels in $pp$ collisions at
$\sqrt{s}=7$ \si{\tera\electronvolt} with the ATLAS detector (2012)
%
\item[] \href{http://iopscience.iop.org/1742-6596/513/3/032102}{\textbf{J. Phys. 2014: Conf. Ser. 513 032102 (2014)}},
A Salvucci et al., PROOF-based analysis on the ATLAS Grid facilities: first
experience with the PoD/PanDa plugin (2014)
%
\item[] \href{http://pos.sissa.it/cgi-bin/reader/conf.cgi?confid=180}{\textbf{Pos. EPS-HEP 2013 121 (2013)}},
A. Salvucci, The new Higgs particle in the $H\to ZZ^{*}\to 4\ell$ searches with
the ATLAS detector (2013)
%
\item[] \href{https://www.sif.it/riviste/sif/ncc/econtents/2013/036/06/article/12}{\textbf{Il Nuovo Cimento Vol. 36 C, N. 6 (2013)}},
A Salvucci, Recent Higgs $\to ZZ^{(∗)}\to  4l$ results with the ATLAS experiment
(2013)
%
\item[] \href{http://www.epj-conferences.org/articles/epjconf/abs/2012/10/epjconf_hcp2012_12039/epjconf_hcp2012_12039.html}{\textbf{EPJ Web of Conferences Volume 28, 12039 (2012)}},
A. Salvucci, Measurement of muon momentum resolution of the ATLAS detector
(2012)
%
\item[] ATLAS author since February 2009: more than 700 pubblications within
the ATLAS Collaboration,\\
  {\scriptsize{\href{http://inspirehep.net/search?ln=en&p=salvucci&of=hb&action_search=Search}{http://inspirehep.net/search?ln=en\&p=salvucci\&of=hb\&action\_search=Search}}}
%
\end{itemize}
}

\end{cvinterests}


%-----------------------------------------------------------------------
%	SECTION TITLE
%-----------------------------------------------------------------------
\cvsection{{Relevant ATLAS CONF notes}}

%-----------------------------------------------------------------------
%	SECTION CONTENT
%-----------------------------------------------------------------------
\begin{cvinterests}

%------------------------------------------------
\cvinterest{
\vspace{-1.8em}
\begin{itemize}[labelwidth=0.05in,align=right,leftmargin=!,labelsep=0pt,
itemsep=0.0em]
%
\item[] \href{https://atlas.web.cern.ch/Atlas/GROUPS/PHYSICS/CONFNOTES/ATLAS-CONF-2018-018/}{\textbf{ATLAS-CONF-2018-018}},
Measurements of the Higgs boson production, fiducial and differential cross
sections in the $4\ell$ decay channel at$\sqrt{s}=13$ \si{\tera\electronvolt}
with the ATLAS detector (2018)
%
\item[] \href{https://atlas.web.cern.ch/Atlas/GROUPS/PHYSICS/CONFNOTES/ATLAS-CONF-2017-032/}{\textbf{ATLAS-CONF-2017-032}},
Measurement of inclusive and differential fiducial cross sections in the
$H\to ZZ^{*}\to 4\ell$ decay channel at \SI{13}{\tera\electronvolt} with the
ATLAS detector (2017)
%
\item[] \href{https://atlas.web.cern.ch/Atlas/GROUPS/PHYSICS/CONFNOTES/ATLAS-CONF-2016-079/}{\textbf{ATLAS-CONF-2016-079}},
Study of the Higgs boson properties and search for high-mass scalar resonances
in the $H\to ZZ^{∗}\to 4\ell$ decay channel at $\sqrt{s}=13$
\si{\tera\electronvolt} with the ATLAS detector (2016 - Editor)
%
\item[] \href{https://atlas.web.cern.ch/Atlas/GROUPS/PHYSICS/CONFNOTES/ATLAS-CONF-2015-072}{\textbf{ATLAS-CONF-2015-072}},
Search for beyond the Standard Model phenomena in $e\mu$ final states in $pp$
collisions at $\sqrt{s}=13$ \si{\tera\electronvolt} with the ATLAS detector
(2015 - Editor)
%
\item[] \href{https://atlas.web.cern.ch/Atlas/GROUPS/PHYSICS/CONFNOTES/ATLAS-CONF-2015-059/}{\textbf{ATLAS-CONF-2015-059}},
Measurements of the Higgs boson production cross section at 7, 8 and 13
\si{\tera\electronvolt} centre-of-mass energies and search for new physics at 13
\si{\tera\electronvolt} $H\to ZZ^{∗}\to 4\ell$ final state with the ATLAS
detector (2015)
%
\item[] \href{https://atlas.web.cern.ch/Atlas/GROUPS/PHYSICS/CONFNOTES/ATLAS-CONF-2015-008/}{\textbf{ATLAS-CONF-2015-008}},
Combination of the Higgs Boson Spin and Parity Analyses of the Higgs Boson in
the $H\to ZZ^{*}\to 4\ell$, $H\to WW\to\ell\nu\ell\nu$ and $H\to\gamma\gamma$
final states (2015)
%
\item[] \href{https://atlas.web.cern.ch/Atlas/GROUPS/PHYSICS/CONFNOTES/ATLAS-CONF-2014-044/}{\textbf{ATLAS-CONF-2014-044}},
Inclusive and differential fiducial cross sections of Higgs boson production
measured in the $H\to ZZ^{*}\to 4\ell$ decay channel using $\sqrt{s}=8$
\si{\tera\electronvolt} $pp$ collision data recorded by the ATLAS detector
(2014)
%
\item[] \href{https://cds.cern.ch/record/1580207}{\textbf{ATLAS-CONF-2013-088}},
Preliminary results on the muon reconstruction efficiency, momentum resolution,
and momentum scale in ATLAS 2012 $pp$ collision data (2013)
%
\item[] \href{https://atlas.web.cern.ch/Atlas/GROUPS/PHYSICS/CONFNOTES/ATLAS-CONF-2013-040/}{\textbf{ATLAS-CONF-2013-040}},
Study of the spin of the new boson with up to \SI{25}{\per\femto\barn} of ATLAS
data (2013)
%
\item[] \href{https://atlas.web.cern.ch/Atlas/GROUPS/PHYSICS/CONFNOTES/ATLAS-CONF-2013-013/}{\textbf{ATLAS-CONF-2013-013}},
Measurements of the properties of the Higgs-like boson in the four lepton decay
channel with the ATLAS detector using \SI{25}{\per\femto\barn} of proton-proton
collision data (2013)
%
\item[] \href{https://atlas.web.cern.ch/Atlas/GROUPS/PHYSICS/CONFNOTES/ATLAS-CONF-2012-170/}{\textbf{ATLAS-CONF-2012-170}},
An update of combined measurements of the new Higgs-like boson with high mass
resolution channels (2012)
%
\item[] \href{https://atlas.web.cern.ch/Atlas/GROUPS/PHYSICS/CONFNOTES/ATLAS-CONF-2012-169/}{\textbf{ATLAS-CONF-2012-169}},
Updated results and measurements of properties of the new Higgs-like particle
in the four lepton decay channel with the ATLAS detector (2012)
%
\item[] \href{https://atlas.web.cern.ch/Atlas/GROUPS/PHYSICS/CONFNOTES/ATLAS-CONF-2012-092/}{\textbf{ATLAS-CONF-2012-092}},
Observation of an excess of events in the search for the Standard Model Higgs
boson in the $H\to ZZ^{*}\to 4\ell$ channel with the ATLAS detector (2012)
%
\item[] \href{https://atlas.web.cern.ch/Atlas/GROUPS/PHYSICS/CONFNOTES/ATLAS-CONF-2012-019/}{\textbf{ATLAS-CONF-2012-019}},
An update to the combined search for the Standard Model Higgs boson with the
ATLAS detector at the LHC using up to \SI{4.9}{\per\femto\barn} of pp collision
data at $\sqrt{s}=7$ \si{\tera\electronvolt} (2012)
%
\item[] \href{https://atlas.web.cern.ch/Atlas/GROUPS/PHYSICS/CONFNOTES/ATLAS-CONF-2011-046/}{\textbf{ATLAS-CONF-2011-046}},
ATLAS Muon Momentum Resolution in the First Pass Reconstruction of the 2010
$p$-$p$ Collision Data at $\sqrt{s}=7$ \si{\tera\electronvolt} (2011)
%
\end{itemize}
}
\end{cvinterests}

%
%%-----------------------------------------------------------------------
%%	SECTION TITLE
%%-----------------------------------------------------------------------
%\cvsection{{Relevant Internal notes}}
%
%%-----------------------------------------------------------------------
%%	SECTION CONTENT
%%-----------------------------------------------------------------------
%\begin{cvinterests}
%
%%------------------------------------------------
%\cvinterest{
%\vspace{-1.8em}
%\begin{itemize}[labelwidth=0.05in,align=right,leftmargin=!,labelsep=0pt,
%itemsep=0.0em]
%%
%\item[] \href{https://cds.cern.ch/record/2301918}{\textbf{ATL-COM-PHYS-2018-050}},
%Measurements of the Higgs boson production, fiducial and differential cross
%sections in the $4\ell$ decay channel at 13  \si{\tera\electronvolt} with the
%ATLAS detector (Editor)
%%
%\item[] \href{https://cds.cern.ch/record/2231597}{\textbf{ATL-COM-PHYS-2016-1605}},
%Measurement of the fiducial, differential and total Higgs production cross
%sections in the $H\to ZZ^{∗}\to 4\ell$ final state from proton–proton collisions
%at $\sqrt{s}=13$ \si{\tera\electronvolt} (Editor)
%%
%\item[] \href{https://cds.cern.ch/record/2229246}{\textbf{ATL-COM-PHYS-2016-1571}},
%Search for new phenomena in different-flavour high mass dilepton final states in
%$pp$ collisions at $\sqrt{s}=13$ \si{\tera\electronvolt} with the ATLAS detector
%(Editor)
%%
%\item[] \href{https://cds.cern.ch/record/2135369}{\textbf{ATL-COM-PHYS-2016-205}},
%Search for a BSM phenomena decaying into $\ell\ell^{\prime}$ final states in $pp$
%collisions at $\sqrt{s}=13$ \si{\tera\electronvolt} with the ATLAS detector
%(Editor)
%%
%\item[]\href{https://cds.cern.ch/record/1995523}{\textbf{ATL-COM-PHYS-2015-160}},
%Support documentation for study of the spin and parity of the Higgs boson in
%HVV decays with the ATLAS detector
%%
%\item[] \href{https://cds.cern.ch/record/2108986}{\textbf{ATLAS-COM-CONF-2015-082}},
%Search for beyond the Standard Model phenomena in $e\mu$ final states in $pp$
%collisions at $\sqrt{s}=13$ \si{\tera\electronvolt} with the ATLAS detector
%(Editor)
%%  
%\item[] \href{https://cds.cern.ch/record/1648266}{\textbf{ATL-COM-PHYS-2014-101}},
%Measurement of the Higgs boson spin, parity, and HZZ tensor structure in the
%four-lepton decay channel with the ATLAS detector
%%
%\item[] \href{https://cds.cern.ch/record/1692617}{\textbf{ATL-COM-PHYS-2014-256}},
%Measurements of the properties of the Higgs boson in the four lepton decay
%channel with the ATLAS detector using \SI{25}{\per\fb} of proton-proton
%collision data
%%
%\item[] \href{https://cds.cern.ch/record/1696343}{\textbf{ATL-COM-MUON-2014-025}},
%Muon reconstruction performances of the ATLAS detector during Run 1
%%
%\item[] \href{https://cds.cern.ch/record/1602621}{\textbf{ATL-COM-SOFT-2013-058}},
%PROOF-based analysis on the ATLAS Grid facilities: first experience with the
%PoD/PanDa plugin
%%
%\item[] \href{https://cds.cern.ch/record/1517018}{\textbf{ATLAS-COM-CONF-2013-018}},
%Measurements of the properties of the Higgs-like boson in the four lepton decay
%channel with the ATLAS detector using \SI{25}{\per\femto\barn} of proton-proton
%collision data
%%
%\item[] \href{https://cds.cern.ch/record/1497267}{\textbf{ATLAS-COM-CONF-2012-204}},
%Observation of an excess of events in the search for the Standard Model Higgs
%boson in the $H\to ZZ^{∗}\to 4\ell$ channel with the ATLAS detector
%%
%\item[] \href{https://cds.cern.ch/record/1455951}{\textbf{ATLAS-COM-CONF-2012-106}},
%Observation of an excess of events in the search for the Standard Model Higgs
%boson in the $H\to ZZ^{∗}\to 4\ell$ channel with the ATLAS detector.
%% 
%\item[] \href{https://cds.cern.ch/record/1361300}{\textbf{ATL-COM-PHYS-2011-751}},
%Total and differential $W\to\ell\nu$ and $Z\to\ell\ell$ cross-sections
%measurements in proton-proton collisions at $\sqrt{s}=7$ \si{\tera\electronvolt}
%with the ATLAS Detector
%%
%\item[] \href{https://cds.cern.ch/record/1322424}{\textbf{ATLAS-COM-CONF-2011-003}},
%Muon Momentum Resolution in First Pass Reconstruction of $pp$ Collision Data
%Recorded by ATLAS in 2010
%%
%\item[] \href{https://cds.cern.ch/record/1191139}{\textbf{ATL-PHYS-INT-2009-067}},
%Impact of Inner Detector and Muon Spectrometer Misalignments on Physics
%%
%\end{itemize}
%}
%\end{cvinterests}

\clearpage
%-----------------------------------------------------------------------
%	SECTION TITLE
%-----------------------------------------------------------------------
\cvsection{{Details of Research Experience}}

%-----------------------------------------------------------------------
%	SECTION CONTENT
%-----------------------------------------------------------------------
\begin{cvinterests}

%------------------------------------------------

\cvinterest
{
\textbf{Exotics searches}\\
\vspace{-1.8em}
\begin{itemize}[labelwidth=0.05in,align=right,leftmargin=!,labelsep=0pt,
itemsep=0.0em]
\item[] \textsc{Search for new possible heavy neutral lepton-flavor violating
(LFV) dilepton final states}\\
Very strong contribution in the search for new possible heavy neutral
lepton-flavor violating dilepton final states, as analysis contact, editor
of publications and main analyser. A lepton-flavour violating decay would be a
clear indication of new physics. Moreover, processes with flavor-violating
dilepton decays have a clear experimental signature and a low SM background.
Given this, the easiest way to look for such process is to search into a high
mass regime and look for events with one different-flavor lepton pair
($e\mu$, $e\tau$, $\mu\tau$) in the final state. Different Beyond Standard
Model (BSM) models have been considered in this search: the LFV $Z^{\prime}$,
the R-parity Violation (RPV) SuperSymmetric (SUSY) $\tilde{\nu}_{t}$, both
predicting a high mass resonance, and Quantum Black Hole (QBH) production.\\
The first round of this search has been published using
\SI{3.2}{\per\femto\barn} of $pp$ collision data at 13 \si{\tera\electronvolt}
considering only electrons and muons in the final state
(\href{https://atlas.web.cern.ch/Atlas/GROUPS/PHYSICS/CONFNOTES/ATLAS-CONF-2015-072}{ATLAS-CONF-2015-072}).
A second round has been published using the same amount of $pp$ data and
including also taus in the final state (\href{https://link.springer.com/article/10.1140/epjc/s10052-016-4385-1}{Eur. Phys. J. C 76 (2016) 541}).
The last round aims at publishing a paper using \SI{36.1}{\per\femto\barn} of
$pp$ collision data recorded between 2015 and 2016: the paper has been already submitted to
PRL journal (\href{https://arxiv.org/abs/1807.06573}{arXiv:1807.06573 [hep-ex]}).\\
For this search I developed the official analysis framework, which provides the
main object and event selection and the SM background processes estimation:
it is being used since Summer 2015. I optimised the event
selection in order to get the highest background rejection and signal
significance. I also studied the background processes arising from jets
mis-reconstructed as leptons ($W$+jets and multi-jet processes), which have a
key role in final states with taus, being up to 50\% of the total expected SM
background. Moreover I evaluated the relevant systematic uncertainties
(theoretical and experimental ones) on the background estimates.
%
\item[] \textsc{Search for new possible heavy neutral gauge bosons decaying in
muon pairs}\\
During my master thesis I studied the discovery potential of the ATLAS
experiment and the production and decay properties of a possible new gauge
heavy boson, $Z^{\prime}$, decaying in a pair of muons. In order to understand
the detector capability in reconstructing such decays, I studied the
performance of the ATLAS tracking system, such as the muon momentum resolution
and tracking and trigger efficiencies, highlighting a large decrease of
efficiency in the presence of strong detector misalignments. Such effect, due to
the failure in matching Inner Detector (ID) and Muon Spectrometer (MS) tracks,
was fixed in the main muon reconstruction code and reported in an ATLAS internal
note (\href{https://cds.cern.ch/record/1191139}{ATL-PHYS-INT-2009-067}).
The ATLAS discovery potential has been studied focusing on the $Z^{\prime}$ boson
predicted by a simple extension of the SM - the Sequential Standard Model (SSM).
The statistical method of a binned likelihood ratio was used for this study.
\end{itemize}
%
\textbf{Standard Model Higgs boson search and property measurements}\\
\vspace{-1.8em}
\begin{itemize}[labelwidth=0.05in,align=right,leftmargin=!,labelsep=0pt,
itemsep=0.0em]
\item[] \textsc{Higgs boson search and discovery}\\
Strong contribution to the Higgs boson search analysis in the the four lepton
decay channel.\\
In the beginning of 2012, I strongly contributed to the kinematic selection
optimisation, fine tuned in the 120--130 \si{\giga\electronvolt} mass region.
The optimisation has been done in order to increase the signal-background
discrimination power and to reach a higher sensitivity to Higgs boson signal
events. The kinematic variables considered were the opposite-charge dilepton
invariant mass closer to the PDG $Z$ mass, the other opposite-charge dilepton
invariant mass and the leptons transverse momenta. I contributed to the
background estimation of the irreducible background, verifying the SM
prediction of continuum $ZZ$ process through dedicated studies and using
different MC generators. I also contributed to the reducible background
estimation ($Z$+jets and $t\bar{t}$), performed in control regions where the
$b\bar{b}$ or $t\bar{t}$ contributions are enhanced
(\href{https://atlas.web.cern.ch/Atlas/GROUPS/PHYSICS/CONFNOTES/ATLAS-CONF-2012-092/}{ATLAS-CONF-2012-092},
\href{http://www.sciencedirect.com/science/article/pii/S037026931200857X}{Phys. Lett. B 716 (2012) 1-29}).
%
\item[] \textsc{Higgs boson property measurement}\\
Key contribution to the Higgs boson spin/CP measurement and participation
to the first Higgs boson mass and couplings measurement in the context of the
four lepton decay channel.\\
After the Higgs boson discovery I started the spin and parity properties
measurement. I produced and validated the Monte Carlo signal samples under
different spin and parity hypotheses. The final state of the ``golden channel''
is well defined from the four reconstructed leptons: this means that the
spin/parity properties can be exploited using the angular and kinematic
distributions. I developed a machine learning algorithm based on the Toolkit
for Multivariate Analysis (TMVA) with the Boosted Decision Tree (BDT) algorithm.
The separation between states with different spin
and parity has been achieved by using a multivariate discriminant, built for
each spin/parity hypothesis using seven sensitive variables to the underlying
spin and parity: the masses of the $Z$ bosons, one production angle and four
decay angles. By using only the events in the $m_{4l}$ window 115-130
\si{\giga\electronvolt}, the exclusion of a spin/parity hypothesis was
achieved by comparing the response shape of the BDT discriminants calculated
for the signal and background samples to those observed in data
(\href{https://atlas.web.cern.ch/Atlas/GROUPS/PHYSICS/CONFNOTES/ATLAS-CONF-2012-169/}{ATLAS-CONF-2012-169}, 
\href{http://www.sciencedirect.com/science/article/pii/S0370269313006369}{Phys. Lett. B 726 (2013), pp. 88-119}).\\
At the end of the RUNI data taking, the whole analysis has been updated using
the full RUNI $pp$ recorded data (\SI{25}{\per\femto\barn}). I performed a
further optimisation of the selection criteria, such as the lepton quadruplet
selection criteria, in order to reduce the amount of background events and
increase the signal significance. Given the Higgs boson spin/CP nature, I
strongly contributed to the further improvement to the signal versus $ZZ*$
background discrimination in order to increase the resolution of the mass
measurement. This was done developing a multivariate kinematic discriminant,
based on the $p_{T}$ and $\eta$ of the four-lepton final state and the
Leading Order Matrix Element information, taking the advantage of the spin/CP
hypothesis of the Higgs boson. Moreover, I contributed to the development of
a new multi-dimensional fit procedure to extract the mass value of the Higgs
boson and, consequently, the couplings
(\href{https://atlas.web.cern.ch/Atlas/GROUPS/PHYSICS/CONFNOTES/ATLAS-CONF-2013-013/}{ATLAS-CONF-2013-013},
\href{http://journals.aps.org/prd/abstract/10.1103/PhysRevD.90.052004}{Phys. Rev. D 90, 052004 (2014)},
\href{https://journals.aps.org/prd/abstract/10.1103/PhysRevD.91.012006}{Phys. Rev. D 91, 012006 (2015)}).
%
\item[] \textsc{Tensor structure and Higgs boson Spin/CP measurement update}\\
Contribution in cross-checking the results by developing an independent code
for toys generation and in editing the supporting documentation for the
published paper.\\
This study has been done using the full LHC RUNI LHC data and the three bosonic
channels: $H\to\gamma\gamma$, $H\to ZZ^*\to4l$ and $H\to WW^*\to l\nu l\nu$.
Concerning the spin measurement, several alternative models have been included
for the spin 2 hypothesis, with universal and non-universal couplings to
fermions and bosons, with respect to the previous published analysis
(\href{http://www.sciencedirect.com/science/article/pii/S0370269313006369}{Phys. Lett. B 726 (2013), pp. 88-119}).
Moreover, the tensor structure of the HVV interaction in the spin-0
hypothesis has been investigated by using the $H\to ZZ^*\to4l$ and
$H\to WW^*\to l\nu l\nu$ decay modes
(\href{https://atlas.web.cern.ch/Atlas/GROUPS/PHYSICS/CONFNOTES/ATLAS-CONF-2015-008/}{ATLAS-CONF-2015-008},
\href{https://link.springer.com/article/10.1140/epjc/s10052-015-3685-1}{Eur. Phys. J. C75 (2015) 476}).
%
\item[] \textsc{Higgs boson fiducial and differential cross section
measurement}:\\
Strong contribution to the Higgs boson fiducial and differential cross section
measurement in the context of the four lepton decay channel.\\
During the LHC RUNII data taking, the analysis aiming at measuring the Higgs
boson cross section has been performed in several rounds, each one published or
aiming to be published as CONF note or paper. My contribution started in Spring
2016, when I performed the cross-check analysis of the fiducial cross section
measurement published using \SI{14.8}{\per\femto\barn} of 13 \si{\tera\electronvolt}
data as a CONF note (\href{https://atlas.web.cern.ch/Atlas/GROUPS/PHYSICS/CONFNOTES/ATLAS-CONF-2016-079/}{ATLAS-CONF-2016-079}).
I then became one of the main analyser and editor of the supporting
documentation in the next analysis round. One of the main goal of this analysis
round was to have a model-independent result that is sensitive to a possible
deviation from the Standard Model. The cross section measurement has been
made within a phase-space defined to mimic the one used for the reconstruction
of Higgs boson decays in the four lepton final state. The fiducial cross section
measurements have been performed using a method based on a fit of the invariant
mass distribution. In addition to the previous published analysis with
\SI{13}{\tera\electronvolt} data, the differential cross section measurements
have been included and performed on seven variables of interest, which either
describe the Higgs kinematics ($p_{T}$, $y$, $\cos\theta^{∗}$, $m_{12}$ and
$m_{34}$) or are sensitive to the details of the Higgs boson production
($N_{\mathrm{jets}}$, $p_{T}^{lj}$, $m_{jj}$, $\Delta\phi_{jj}$, $\Delta\eta_{jj}$).
Also the first double-differential cross section measurements
($p_{T}$ versus $N_{\mathrm{jets}}$ and $m_{12}$ versus $m_{34}$) have been performed.
Given the sensitivity of $m_{12}$ versus $m_{34}$ kinematic plane to Beyond Standard
Model (BSM) couplings, limits on modified Higgs boson interactions within the
framework of pseudo-observables have been set. The results have been published
with \SI{36.1}{\per\femto\barn} of 13 \si{\tera\electronvolt} data as a CONF
note (\href{https://atlas.web.cern.ch/Atlas/GROUPS/PHYSICS/CONFNOTES/ATLAS-CONF-2017-032/}{ATLAS-CONF-2017-032})
and a paper (\href{https://link.springer.com/article/10.1007/JHEP10(2017)132}{JHEP 10 (2017) 132}).\\
A second round of the analysis has been published as a CONF note
(\href{https://atlas.web.cern.ch/Atlas/GROUPS/PHYSICS/CONFNOTES/ATLAS-CONF-2018-018/}{ATLAS-CONF-2018-018})
containing the results with \SI{80}{\per\femto\barn} of 13 \si{\tera\electronvolt}.\\
Currently I am involved in the next analysis round which aims to publish results
with full RUN2 data. I'm covering the role of analysis contact and main
analyser. I expect to improve the unfolding method employed (matrix inversion
instead of the bin-by-bin unfolding) and to consider some additional sensitive
variables. The change of the unfolding method requires a consistent
optimisation of the analysis framework: the efficiency matrix is built from MC
simulation, connecting the events kinematic at reconstruction level with the
one at particle level. Instead of making use of the truth matching tools
provided by ROOT, I performed a selection based on C++ pointers, storing just
the needed information on the shared memory. As a result, the analysis
framework gets faster by a factor 2 using a very low amount (around 500MB) of
memory when running on more than 15M events.
%
\end{itemize}
%
\textbf{Muon reconstruction and performance}\\
\vspace{-1.8em}
\begin{itemize}[labelwidth=0.05in,align=right,leftmargin=!,labelsep=0pt,
itemsep=0.0em]
\item[] Strong contribution to the measurement of the muon reconstruction
performance of the ATLAS detector. I evaluated the muon momentum resolution of
the ATLAS tracking system  as a function of the muon $p_{T}$ by using
data-driven techniques. The $Z$ boson decay into muons has been used. Since
the ATLAS detector is equipped with an Inner Detector (ID) and a Muon
Spectrometer (MS), combined muon tracks have been considered. The muon momentum
resolution has been obtained performing a ``global'' fit procedure in which
both the information of the ID and the MS are used. The input quantities of
the fit technique are the reconstructed di-muon invariant mass peak at the $Z$
pole, the difference between the independent momentum measurements of the ID
and the MS and external constraints on MS alignment and multiple scattering in
the ID and the MS. The final values of the muon momentum parameters have been
obtained using an iterative $\chi^{2}$ minimisation fit. The momentum resolution
corrections for physics analysis have been also derived and provided for physics
analyses making an official ATLAS analysis tool
(\texttt{MuonMomentumCorrections}). This work has been published in a paper
(\href{http://link.springer.com/article/10.1140\%2Fepjc\%2Fs10052-014-3130-x}{Eur. Phys. J. C74 (2014) 3130})
and two conference notes (\href{https://cds.cern.ch/record/1580207}{ATLAS-CONF-2013-088},
\href{https://atlas.web.cern.ch/Atlas/GROUPS/PHYSICS/CONFNOTES/ATLAS-CONF-2011-046/}{ATLAS-CONF-2011-046}).
\end{itemize}
%
\textbf{Monte Carlo generation and validation}\\
\vspace{-1.8em}
\begin{itemize}[labelwidth=0.05in,align=right,leftmargin=!,labelsep=0pt,
itemsep=0.0em]
\item[] \textsc{Higgs MC Production Manager}\\
I am currently covering the role of Higgs MC Production Manager since November
2016. I'm responsible for Monte Carlo production, validation and submission
of physics processes on behalf of the ATLAS Higgs group. Among all duties such
role implies, the most important one is to plan and effectively apply the best
strategy to simulate physics processes in agreement with the Higgs analysis
groups, Physics and Modelling group and Physics Coordination. Moreover, I am
responsible for JobOption preparation (python based) and validation, and any
other detail before actual submitting the request to the ATLAS Monte Carlo
production team. I also provide technical support on the Monte Carlo generator,
Parton Shower and PDF choice when needed.
%
\item[] \textsc{Responsible for Monte Carlo Simulation in the HZZ group}\\
In the context of the $H\to ZZ^{∗}\to 4l$ group, I have been responsible for 
the Monte Carlo production, between September 2012 and April 2014. Covering such
role, I configured, validated and planned several MC samples for both search
analysis (continuum ZZ background samples using the Powheg MC generator,
reducible $Z$+jets and $Z$+$bb$ samples using the Alpgen MC generator) and
properties measurements (Higgs signal samples for various spin and parity
hypotheses using the JHU MC generator).\\
Later on, I have been strongly involved in the preparation and optimisation of
MC samples production for the LHC RUNII. In particular I optimised the
production of the continuum ZZ background by using the Powheg on-the-fly
generation and validated the production machinery for the reducible $Z$+$bb$
background samples by using Sherpa MC generator.
\end{itemize}
%
\textbf{Computing and Software}\\
\vspace{-1.8em}
\begin{itemize}[labelwidth=0.05in,align=right,leftmargin=!,labelsep=0pt,
itemsep=0.0em]
\item[] I used the physics analysis framework developed in the context of the 
$H\to ZZ^{∗}\to 4\ell$ analysis to investigate a new parallel computing
framework based on the PROOF system and using the ATLAS Grid facilities.
I investigated the usage of Proof-on-Demand (PoD) to enable PROOF-based analysis
on ATLAS using the developed PoD/PanDA plug-in interface. Data has been
accessed using two different protocols: XRootD and file protocol. This work
has been presented in the 20th International Conference on Computing in High
Energy and Nuclear Physics (CHEP 2013, Amsterdam) and published on the Journal
of Physics (\href{http://iopscience.iop.org/1742-6596/513/3/032102}{J. Phys. 2014: Conf. Ser. 513 032102}).\\
In the context of the combined muon performance, I developed and maintained,
between 2011 and 2012, an official ATLAS analysis tool
(\texttt{MuonMomentumCorrections}), providing muon resolution, smearing
functions and corrections for physics analyses.\\
I also contributed to the development of the official ATLAS Monte Carlo
on-the-fly generation system, focusing on Alpgen and Prophecy MC generators.
\end{itemize}
%
}

\end{cvinterests}

\fi
\ifResearch
\clearpage
%-----------------------------------------------------------------------
%	SECTION TITLE
%-----------------------------------------------------------------------
\cvsection{{Research Interests}}

%-----------------------------------------------------------------------
%	SECTION CONTENT
%-----------------------------------------------------------------------
\begin{cvinterests}

%------------------------------------------------

\cvinterest
{
I am a member of the ATLAS collaboration since February 2008, when I started my
master thesis at the ``Laboratori Nazionali di Frascati'' (LNF) in
Italy. In these almost 10 year of research experience within the ATLAS
experiment, I have been deeply involved in physics analyses, detector
performance studies (mostly concentrated on the muon tracking system) and
Monte Carlo production. On the physics analysis side, I have strongly
contributed to the Standard Model (SM) Higgs boson search and properties
measurement (spin/parity, couplings and cross section) and to the search for
beyond SM signatures, such us the heavy neutral $Z^{\prime}$ boson - decaying
into a pair of muons - and the search for new possible heavy neutral
lepton-flavor violating dilepton final states.\\
%
In the past two years, the ATLAS detector collected around
\SI{80}{\per\femto\barn} of proton-proton collisions data at
\SI{13}{\tera\electronvolt} and by the end of the RUN2 we expect to have a total
integrated luminosity of about \SI{140}{\per\femto\barn}. Therefore, we can
certainly improve the current precision measurements of the Higgs boson
properties and probe for new physics signatures beyond the SM.\\
% Higgs boson cross section and couplings
For instance, by using the full RUN2 data we will be able to improve the
current precision of the Higgs boson differential cross section measurement.
Moreover, by performing this measurement according to a model-independent
approach, it would be possible to probe possible deviations from the SM. In the
context of the four lepton decay channel, various observables describing the
Higgs boson production and decay can be considered. The Higgs boson transverse
momentum can be used to test perturbative QCD calculations, especially when
separated into exclusive jet multiplicities. The magnitude of the cosine of the
decay angle of the leading lepton pair in the four-lepton rest frame with
respect to the beam axis and the invariant mass of the sub-leading lepton pair
can be used to test the spin and parity of the Higgs boson. Moreover, the cross
section measured in the plane defined by the invariant masses of the leading
and sub-leading lepton pairs can be interpreted in the framework of
pseudo-observables, thus allowing to probe anomalous Higgs boson interactions
with leptons and $Z$ bosons.\\
% di-Higgs searches
One of the most important next step is to directly probe and constrain the
couplings of the Higgs boson to the content of the SM. Particularly crucial is
the measurement of the Higgs self-coupling measurement, which will allow to
understand the structure of the electroweak symmetry breaking (EWSB) mechanism.
The self-coupling is one mechanism for Higgs boson pair production, but Higgs
boson pairs can also be produced through other interactions, such as the
Higgs-fermion Yukawa interactions in the SM. These processes are collectively
referred to as non-resonant production and have an extremely small SM
expectation due to destructive interference among diagrams. On the other hand,
physics beyond the SM can potentially enhance the production rate and alter the
event kinematics. For example, in the Minimal Supersymmetric Standard Model
(MSSM), a heavy CP-even neutral Higgs boson $H$ can decay to a pair of lighter
Higgs bosons. The production of such $H$, and its consequent decay $H\to hh$,
would lead to a new resonant process of Higgs boson pair production, in
contrast to the non-resonant $hh$ production predicted by the SM.
Among all possible decay channels, those with $b$-jets in the final state have
the largest branching ratios; in particular, the branching ratios are 33.4\%
and 25\% for the $hh\to bbbb$ and $hh\to bbWW$ processes, respectively.
The ATLAS collaboration has already published searches for pair production of
Higgs bosons (for both resonant and non-resonant production) using RUN1 and
early RUN2 data, observing no significant data excess above the background
expectation. The difficulties can be attributed to the enormous amount of
background events in the final state: for instance, the $hh\to bbbb$ decay
channel is dominated by the QCD background originating from multi-jet
production. In addition, such fully-hadronic final state is particularly
challenging to be triggered. I strongly believe that the improvement of trigger
and $b$-tagging strategies can help to retain more signal and allow novel
reconstruction techniques to achieve a better performance.\\
In addition to physics analyses, I would be strongly interested to take part
into detector performance studies and/or upgrade projects for the ATLAS
detector. During my PhD I played an important role in the Combined Muon
Performance group (MCP). In that occasion, I studied the the muon momentum
resolution of the ATLAS tracking system. I am strongly interested to contribute
to the ATLAS Upgrade Projects (of the new inner tracker, in particular) on which
the Nikhef is involved. Being part of this project not only give me the
opportunity to put my experience at the service of the group, but also broaden
my knowledge on detector instrumentation.
}

\end{cvinterests}

\else
\ifAddInterest
%-----------------------------------------------------------------------
%	SECTION TITLE
%-----------------------------------------------------------------------
\cvsection{{Interests}}

%-----------------------------------------------------------------------
%	SECTION CONTENT
%-----------------------------------------------------------------------
\begin{cvinterests}

%------------------------------------------------

\cvinterest
{Information technology and console games; woodworking; puzzles and Lego;
mushroom hunting.}

\end{cvinterests}

\fi
\fi
\ifAddAnnex
\clearpage
\input{annex}
\fi

%--------------------------------------------------------------------------

\end{document}
