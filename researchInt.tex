%-----------------------------------------------------------------------
%	SECTION TITLE
%-----------------------------------------------------------------------
\cvsection{{Research Interests}}

%-----------------------------------------------------------------------
%	SECTION CONTENT
%-----------------------------------------------------------------------
\begin{cvinterests}

%------------------------------------------------

\cvinterest
{
I am a member of the ATLAS collaboration since February 2008, when I started my
master thesis at the ``Laboratori Nazionali di Frascati'' (LNF) in
Italy. In these almost 10 year of research experience within the ATLAS
experiment, I have been deeply involved in physics analyses, detector
performance studies (mostly concentrated on the muon tracking system) and
Monte Carlo production. On the physics analysis side, I have strongly
contributed to the Standard Model (SM) Higgs boson search and properties
measurement (spin/parity, couplings and cross section) and to the search for
beyond SM signatures, such us the heavy neutral $Z^{\prime}$ boson - decaying
into a pair of muons - and the search for new possible heavy neutral
lepton-flavor violating dilepton final states.\\
%
In the past two years, the ATLAS detector collected around
\SI{80}{\per\femto\barn} of proton-proton collisions data at
\SI{13}{\tera\electronvolt} and by the end of the RUN2 we expect to have a total
integrated luminosity of about \SI{140}{\per\femto\barn}. Therefore, we can
certainly improve the current precision measurements of the Higgs boson
properties and probe for new physics signatures beyond the SM.\\
% Higgs boson cross section and couplings
For instance, by using the full RUN2 data we will be able to improve the
current precision of the Higgs boson differential cross section measurement.
Moreover, by performing this measurement according to a model-independent
approach, it would be possible to probe possible deviations from the SM. In the
context of the four lepton decay channel, various observables describing the
Higgs boson production and decay can be considered. The Higgs boson transverse
momentum can be used to test perturbative QCD calculations, especially when
separated into exclusive jet multiplicities. The magnitude of the cosine of the
decay angle of the leading lepton pair in the four-lepton rest frame with
respect to the beam axis and the invariant mass of the sub-leading lepton pair
can be used to test the spin and parity of the Higgs boson. Moreover, the cross
section measured in the plane defined by the invariant masses of the leading
and sub-leading lepton pairs can be interpreted in the framework of
pseudo-observables, thus allowing to probe anomalous Higgs boson interactions
with leptons and $Z$ bosons.\\
% di-Higgs searches
One of the most important next step is to directly probe and constrain the
couplings of the Higgs boson to the content of the SM. Particularly crucial is
the measurement of the Higgs self-coupling measurement, which will allow to
understand the structure of the electroweak symmetry breaking (EWSB) mechanism.
The self-coupling is one mechanism for Higgs boson pair production, but Higgs
boson pairs can also be produced through other interactions, such as the
Higgs-fermion Yukawa interactions in the SM. These processes are collectively
referred to as non-resonant production and have an extremely small SM
expectation due to destructive interference among diagrams. On the other hand,
physics beyond the SM can potentially enhance the production rate and alter the
event kinematics. For example, in the Minimal Supersymmetric Standard Model
(MSSM), a heavy CP-even neutral Higgs boson $H$ can decay to a pair of lighter
Higgs bosons. The production of such $H$, and its consequent decay $H\to hh$,
would lead to a new resonant process of Higgs boson pair production, in
contrast to the non-resonant $hh$ production predicted by the SM.
Among all possible decay channels, those with $b$-jets in the final state have
the largest branching ratios; in particular, the branching ratios are 33.4\%
and 25\% for the $hh\to bbbb$ and $hh\to bbWW$ processes, respectively.
The ATLAS collaboration has already published searches for pair production of
Higgs bosons (for both resonant and non-resonant production) using RUN1 and
early RUN2 data, observing no significant data excess above the background
expectation. The difficulties can be attributed to the enormous amount of
background events in the final state: for instance, the $hh\to bbbb$ decay
channel is dominated by the QCD background originating from multi-jet
production. In addition, such fully-hadronic final state is particularly
challenging to be triggered. I strongly believe that the improvement of trigger
and $b$-tagging strategies can help to retain more signal and allow novel
reconstruction techniques to achieve a better performance.\\
In addition to physics analyses, I would be strongly interested to take part
into detector performance studies and/or upgrade projects for the ATLAS
detector. During my PhD I played an important role in the Combined Muon
Performance group (MCP). In that occasion, I studied the the muon momentum
resolution of the ATLAS tracking system. I am strongly interested to contribute
to the ATLAS Upgrade Projects (of the new inner tracker, in particular) on which
the Nikhef is involved. Being part of this project not only give me the
opportunity to put my experience at the service of the group, but also broaden
my knowledge on detector instrumentation.
}

\end{cvinterests}
